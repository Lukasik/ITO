\documentclass[12pt,a4paper,titlepage,final]{article}
\usepackage[czech]{babel}
\usepackage[utf8]{inputenc}
\usepackage[bookmarksopen,colorlinks,plainpages=false,urlcolor=blue,unicode]{hyperref}
\usepackage{url}
\usepackage{float}
\usepackage[dvipdf]{graphicx}
\usepackage[top=3.5cm, left=2.5cm, text={17cm, 24cm}, ignorefoot]{geometry}

\begin{document}
\newpage
\pagestyle{empty}

%%%%%%%%%%%%%%%%%%%%%%%%%%%%%%%%%%%%%%%%%%%%%%%%%%%%%%%%%%%%%%%%%%%%%%%%%%%%%%
% titulní strana
\def\author{Lukáš Vokráčko}
\def\email{xvokra00@stud.fit.vutbr.cz}
\def\projname{Semestrální projekt - řešení obvodů}
\input{title.tex}

%%%%%%%%%%%%%%%%%%%%%%%%%%%%%%%%%%%%%%%%%%%%%%%%%%%%%%%%%%%%%%%%%%%%%%%%%%%%%%
% obsah

\tableofcontents

%%%%%%%%%%%%%%%%%%%%%%%%%%%%%%%%%%%%%%%%%%%%%%%%%%%%%%%%%%%%%%%%%%%%%%%%%%%%%%
% textova zprava
\newpage
\pagestyle{plain}
\pagenumbering{arabic}
\setcounter{page}{1}
%%%%%%%%%%%%%%%%%%%%%%%%%%%%%%%%%%%%%%%%%%%%%%%%%%%%%%%%%%%%%%%%%%%%%%%%%%%%%%

\section{Příklad 1}
\subsection{Zadání}
Stanovte napětí na $U_{R7}$ a proud $I_{R7}$. Použijte metodu postupného zjednodušování obvodu.
Hodnoty:\\
\begin{tabular}{|c|c|c|c|c|c|c|c|c|c|}
\hline
Sk. & $U [V]$ & $R_1 [\Omega]$ & $R_2 [\Omega]$ & $R_3 [\Omega]$ & $R_4 [\Omega]$ & $R_5 [\Omega]$  & $R_6 [\Omega]$ & $R_7 [\Omega]$ & $R_8 [\Omega]$\\
\hline
B & 95 & 650 & 730 & 340 & 330 & 410 & 830 & 340 & 220\\
\hline
\end{tabular}

\subsection{Postup řešení}
\begin{enumerate}
\item Vypočítám hodnoty paralelně zapojených rezistorů $R_2, R_3$ a $R_7, R_8$.
\item Odpory $R_1, R_{23}, R_4$ transformuji na hvězdu a vypočítám hodnoty odporů $R_a, R_b, R_c$
\item Vypočítám celkový odpor $R_{EKV}$ \ref{eq1:rekv} zjednodušením schématu \ref{fig1}, které vzniklo transfigurací
\item Vypočítám proud $I$
\item Vypočítám napětí $U_{R7}$ \ref{eq1:ur7} na rezistoru $R_{78}$ a z něj proud $I_{R7}$ \ref{eq1:ir7} na rezistoru $R_7$
\end{enumerate}
\begin{figure}[h]
  \centering
  \includegraphics[height=5cm]{img/fig1.eps}
  \caption{Schéma obvodu po tranfiguraci}
  \label{fig1}
\end{figure}

\newcommand{\prvniRdvatri}{\ensuremath{\frac{R_2*R_3}{R_2+R_3}}}
\newcommand{\prvniRsedmosm}{\ensuremath{\frac{R_7*R_8}{R_7+R_8}}}
\newcommand{\prvniRa}{\ensuremath{\frac{R_1*\prvniRdvatri}{R_1+\prvniRdvatri + R_4}}}
\newcommand{\prvniRb}{\ensuremath{\frac{R_1*R_4}{R_1+\prvniRdvatri + R_4}}}
\newcommand{\prvniRc}{\ensuremath{\frac{R_4*\prvniRdvatri}{R_1+\prvniRdvatri + R_4}}}
\begin{equation}\label{eq1:rekv}
R_{EKV} = \prvniRa + \frac{(\prvniRb + R_5)*(\prvniRc+R_6)}{(\prvniRb+R_5)+(\prvniRc+R_6)} + \prvniRsedmosm
\end{equation}
\[R_{EKV} = 612.1807\Omega\]
\[I = \frac{U}{R_{EKV}}\]
\subsection{Výsledky}
\begin{equation}\label{eq1:ur7}
U_{R7} = I * \prvniRsedmosm = 20.7280V
\end{equation}
\begin{equation}\label{eq1:ir7}
I_{R7} = \frac{U_{R7}}{R_7} = 0.0609A
\end{equation}


\section{Příklad 2}
\subsection{Zadání}
Stanovte napětí $U_{R5}$ a proud $I_{R5}$. Použijte metodu Theveninovy věty.\\
Hodnoty:\\
\begin{tabular}{|c|c|c|c|c|c|c|}
\hline
Sk. & $U [V]$ & $R_1 [\Omega]$ & $R_2 [\Omega]$ & $R_3 [\Omega]$ & $R_4 [\Omega]$ & $R_5 [\Omega]$\\
\hline
F & 130 & 350 & 600 & 195 & 320 & 280\\
\hline
\end{tabular}
\subsection{Postup řešení}
\begin{enumerate}
\item Odpojení odporu $R_5$
\item Transfigurace rezistorů $R_1, R_2, R_3$ na hvězdu \ref{fig2}
\item Překreslení obvodu \ref{fig3} pro výpočet celkového odporu $R_i$ \ref{eq2:ri}
\item Výpočet proudu $I$ \ref{eq2:i} protékajícího celým obvodem
\item Nakreslení náhradního schématu \ref{fig4}
\item Výpočet $U_i$\ref{eq2:ui}
\item Výpočet $I_{R5} \ref{eq2:ir5}, U_{R5} \ref{eq2:ur5}$
\end{enumerate}

\begin{figure}[h]
  \centering
  \includegraphics[height=5cm]{img/fig2.eps}
  \caption{Schéma obvodu po tranfiguraci}
  \label{fig2}
\end{figure}

\begin{figure}[h]
  \centering
  \includegraphics[height=5cm]{img/fig3.eps}
  \caption{Překreslený obvod}
  \label{fig3}
\end{figure}

\begin{figure}[h]
  \centering
  \includegraphics[height=5cm]{img/fig4.eps}
  \caption{Náhradní obvod}
  \label{fig4}
\end{figure}

\newcommand{\druhyRa}{\ensuremath{\frac{R_1*R_2}{R_1+ R_2 + R_3}}}
\newcommand{\druhyRb}{\ensuremath{\frac{R_2*R_3}{R_1+ R_2 + R_3}}}
\newcommand{\druhyRc}{\ensuremath{\frac{R_1*R_3}{R_1+ R_2 + R_3}}}
\begin{equation}\label{eq2:ri}
R_i = \druhyRb + \frac{(\druhyRc+R_4)*\druhyRa}{\druhyRc+R_4+\druhyRa}
\end{equation}
\[R_i = 225.8434\Omega\]

\begin{equation}\label{eq2:i}
I = \frac{U}{\druhyRa+R_4+\druhyRc}
\end{equation}

\begin{equation}\label{eq2:ui}
U_i = I*(\druhyRc+R_4)
\end{equation}
\subsection{Výsledky}
\begin{equation}\label{eq2:ir5}
I_{R5} = \frac{U_i}{R_i+R_5} = 0.1732A
\end{equation}

\begin{equation}\label{eq2:ur5}
U_{R5} = R_5 * I_{R5} = 48.5178V
\end{equation}


\section{Příklad 3}
\subsection{Zadání}
Stanovte napětí $U_{R4}$ a proud $I_{R4}$. Použijte metodu uzlových napětí ($U_A, U_B, U_C$).\\
Hodnoty:\\
\begin{tabular}{|c|c|c|c|c|c|c|c|c|c|}
\hline
Sk. & $U_1 [V]$ & $U_2[V]$ & $I[A]$ & $R_1 [\Omega]$ & $R_2 [\Omega]$ & $R_3 [\Omega]$ & $R_4 [\Omega]$ & $R_5 [\Omega]$  & $R_6 [\Omega]$\\
\hline
E & 135 & 55 & 0.65 & 520 & 420 & 520 & 420 & 215 & 305\\
\hline
\end{tabular}
\subsection{Postup řešení}
\begin{enumerate}
\item Zápis rovnic pro proudy $I_{R1}-I_{R6}$ \ref{eq3:proudy}
\item Vytvoření rovnic pro uzlová napětí $A - C$ \ref{eq3:napeti}
\item Řešení soustavy 3 rovnic se 3 neznámými $U_A, U_B, U_C$
\item Výpočet proudu $I_{R4}$ \ref{eq3:ir4}
\end{enumerate}
\begin{equation}\label{eq3:proudy}
I_{R1} = \frac{U_A}{R_1}
\end{equation}
\[I_{R2} = \frac{U_A + U_B + U_1}{R_2}\]
\[I_{R3} = \frac{U_A-U_B}{R_3}\]
\[I_{R4} = \frac{U_C}{R_4}\]
\[I_{R5} = \frac{U_B-U_C}{R_5}\]
\[I_{R6} = \frac{U_2+U_C-U_B}{R_6}\]

\begin{equation}\label{eq3:napeti}
A: I+I_{R2}-I_{R3}-I_{R1} = 0
\end{equation}
\[B:  I_{R3}-I_{R2}+I_{R6}-I_{R5} = 0\]
\[C: I_{R5}-I_{R6}-I_{R4} = 0\]

\begin{equation}\label{eq3:ir4}
U_{R4} = U_C
I_{R4} = \frac{U_C}{R_4}
\end{equation}

\subsection{Výsledky}
\[U_{R4} = U_C = 77.8185V\]
\[I_{R4} = 0.1852A\]

\section{Příklad 4}
\subsection{Zadání}
Pro napájecí napětí platí: $u=U*sin(2\pi f t)$. Ve vztahu pro napětí na cívce: 
$u_L = U_L*sin(2\pi f t+\varphi_L)$ určete $|U_L|$ a $\varphi_L$. Použijte metodu zjednodušování obvodu.\\
Hodnoty:\\
\begin{tabular}{|c|c|c|c|c|c|c|c|c|c|c|}
\hline
Sk. & $U [V]$ & $R_1 [\Omega]$ & $R_2 [\Omega]$ & $R_3 [\Omega]$ & $L [mH]$ & $C_1 [\mu F]$  & $C_2 [\mu F]$ & $f[Hz]$\\
\hline
B & 35 & 160 & 220 & 270 & 480 & 440 & 170 & 85
\\
\hline
\end{tabular}
\subsection{Postup řešení}
\begin{enumerate}
\item Výpočet celkové impedance obvodu \ref{eq4:impedance}
\item Výpočet proudu $i_L$ na větvi s cívkou \ref{eq4:proud}
\item Výpočet napětí $u_L$ na cívce $L$ \ref{eq4:ul}
\item Výpočet $|u_L|$ \ref{eq4:abs}
\item Výpočet $\varphi_L$ \ref{eq4:uhel}
\end{enumerate}
\newcommand{\ctvrtyImpCjedna}{\ensuremath{\frac{1}{2\pi f C_1 j}}}
\newcommand{\ctvrtyImpCdva}{\ensuremath{\frac{1}{2\pi f C_2 j}}}
\newcommand{\ctvrtyImpL}{\ensuremath{2\pi f L j}}
\newcommand{\ctvrtyParL}{\ensuremath{\frac{(\ctvrtyImpCdva+R_2)*(\ctvrtyImpL+R_1)}{(\ctvrtyImpCdva+R_2)+(\ctvrtyImpL+R_1)}}}
\begin{equation}\label{eq4:impedance}
Z = \ctvrtyImpCjedna + \frac{\ctvrtyParL * R_3}{\ctvrtyParL + R_3}\Omega
\end{equation}
\begin{equation}\label{eq4:proud}
i_L = \frac{U - \frac{U}{Z}*\ctvrtyImpCjedna}{\ctvrtyImpL+R_1}A
\end{equation}
\begin{equation}\label{eq4:ul}
u_L = i_L*\ctvrtyImpL V
\end{equation}
\begin{equation}\label{eq4:abs}
|U_L| = \sqrt{Im(u_C)^2+Re(u_C)^2} V
\end{equation}
\begin{equation}\label{eq4:uhel}
\varphi_L = arctan(\frac{Im(u_C)}{Re(u_C)})rad
\end{equation}
\subsection{Výsledky}
\[|u_L| = 29.9934V\]
\[\varphi_L = 0.6009rad\]

\section{Příklad 5}
\subsection{Zadání}
Pro napájecí napětí platí: $u_1=U_1*sin(2\pi f t)$, $u_2=U_2*sin(2\pi f t)$.
Ve vztahu pro napění na cívce $L_1:u_{L_1}=U_{L_1}*sin(2\pi f t + \varphi L_1)$ 
určete $|U_{L_1}|$ a $\varphi L_1$. Použijte metodu smyčkových proudů.\\
Hodnoty:\\
\begin{tabular}{|c|c|c|c|c|c|c|c|c|c|}
\hline
Sk. & $U_1 [V]$ & $U_2 [V]$ & $R_1 [\Omega]$ & $R_2 [\Omega]$ & $L_1[mH]$ & $L_2[mH]$ & $C_1 [\mu F]$ & $C_2 [\mu F]$ & $f [Hz]$\\
\hline
F & 20 & 35 & 120 & 100 & 170 & 80 & 150 & 90 & 65\\
\hline
\end{tabular}
\subsection{Postup řešení}
\begin{enumerate}
\item Vyjádření rovnic pro smyčkové proudy \ref{eq5:proudy} podle obrázku \ref{fig:5}
\item Výpočet smyčkových proudů ze soustavy rovnic
\item Výpočet napětí $u_{L1}$ na cívce $L_{1}$ \ref{eq5:ul1}
\item Výpočet absolutní hodnoty $|U_{L1}|$ \ref{eq5:abs}
\item Výpočet $\varphi_{L1}$ \ref{eq5:fi}
\end{enumerate}
\begin{equation}\label{eq5:proudy}
I_a: X_{C1}I_a+R_1I_a+X_{L2}(I_a-I_c)+X_{C2}(I_a-I_b) = U_1
\end{equation}
\[I_b: X_{L1}(I_b-I_c)+X_{C2}(I_b-I_a)+U_1=0\]
\[I_c: X_{L2}(I_c-I_a)+X_{L1}(I_c-I_b)+R_2I_c+U_2=0\]

\begin{equation}\label{eq5:ul1}
U_{L1} = X_{L1}*(I_b-I_c)
\end{equation}

\begin{equation}\label{eq5:abs}
|U_{L1}| = \sqrt{Im(U_{L1})^2+Re(U_{L1})^2}
\end{equation}

\begin{equation}\label{eq5:fi}
\varphi_{L1} = arctan(\frac{Im(U_{L1})}{Re(U_{L1})})-\pi
\end{equation}
\begin{figure}[h]
  \centering
  \includegraphics[height=5cm]{img/fig5.eps}
  \caption{Směry proudů}
  \label{fig:5}
\end{figure}
\subsection{Výsledky}
\[|U_{L1}| = 45.6745V\]
\[\varphi_{L1} = -2.0907rad\]
\section{Příklad 6}
\subsection{Zadání}
Sestavte diferenciální rovnici popisující chování obvodu na obrázku, dále ji upravte dosazením hodnot parametrů. Vypočítejte analytické řešení $u_C=f(t)$.
Proveďte kontrolu výpočtu dosazením do sestavené diferenciální rovnice.\\
Hodnoty:\\
\begin{tabular}{|c|c|c|c|c|c|c|c|c|c|}
\hline
Sk. & $U [V]$ & $C[F]$ & $R[\Omega]$ & $u_c(0) [V]$\\
\hline
E & 12 & 30 & 45 & 6\\
\hline
\end{tabular}
\subsection{Postup řešení}
\begin{enumerate}
\item Vytvoření obecné rovnice \ref{eq6:obecna}
\item Vyjádření $\lambda$ z charakteristické rovnice \ref{eq6:lambda}
\item Dosazení do očekávaného řešení \ref{eq6:ocekavane}
\item Vytvoření $u_c'$ \ref{eq6:ucderivace}
\item Dosazení do obecné rovnice \ref{eq6:obecnadosazeni}
\item Integrace $K'(t)$ \ref{eq6:integrace}
\item Výpočet konstanty dosazením podmínky \ref{eq6:konst}
\item Dosazení do očekávaného řešení \ref{eq6:dosazeni}
\end{enumerate}

\begin{equation}\label{eq6:obecna}
u_c'=\frac{1}{C}*\frac{U-u_c}{R}
\end{equation}
\[1350u_c'+u_c=12\]

\begin{equation}\label{eq6:lambda}
1350\lambda+1=0
\end{equation}
\[\lambda=-\frac{1}{1350}\]

\begin{equation}\label{eq6:ocekavane}
u_c(t)=K(t)e^{-\frac{t}{1350}}
\end{equation}

\begin{equation}\label{eq6:ucderivace}
u_c'(t)=K'(t)e^{-\frac{t}{1350}}+K(t)(-\frac{1}{1350})e^{-\frac{t}{1350}}
\end{equation}

\begin{equation}\label{eq6:obecnadosazeni}
1350K'(t)e^{-\frac{t}{1350}}+1350K(t)(-\frac{1}{1350})e^{-\frac{t}{1350}})+K(t)e^{-\frac{t}{1350}}=12
\end{equation}
\[K'(t)=\frac{12e^{\frac{t}{1350}}}{1350}\]

\begin{equation}\label{eq6:integrace}
K(t) = \frac{1350*12e^{\frac{t}{1350}}}{1350} + c
\end{equation}

\begin{equation}\label{eq6:konst}
6=12+ce^{-\frac{0}{1350}}
\end{equation}
\[c = -6\]

\begin{equation}\label{eq6:dosazeni}
u_c(t)= 12-6e^{-\frac{t}{1350}}
\end{equation}
\subsection{Výsledky}
\[1350u_c'+u_c=12\]
\[u_c(t)= 12-6e^{-\frac{t}{1350}}\]

\section{Přehled výsledků}

\begin{tabular}{|c|c|c|c|}
\hline
Př. & Sk. & \multicolumn{2}{|c|}{Výsledky}\\
\hline
1 & B & $I_{R7} = 0.0609A$ & $U_{R7} = 20.7280V$\\
\hline
2 & F & $I_{R5} = 0.1732A$ & $U_{R5} = 48.5178V$\\
\hline
3 & E & $I_{R4} = 0.1852A$ & $U_{R4} = 77.8185V$\\
\hline
4 & B & $\varphi_L = 0.6009rad$ & $|u_L| = 29.9934V$\\
\hline
5 & F & $\varphi_{L1} = -2.0907rad$ & $|U_{L1}| = 45.6745V$\\
\hline
6 & E & $1350u_c'+u_c=12$ & $u_c(t)= 12-6e^{-\frac{t}{1350}}$\\
\hline
\end{tabular}
\end{document}